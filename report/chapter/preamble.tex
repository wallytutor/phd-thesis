\cleardoublepage\thispagestyle{empty}\vspace*{\fill}%
\begin{flushright}\begin{minipage}{0.7\textwidth}%
		\begin{flushright}\itshape%
			„Glaube“ heißt Nicht-wissen-wollen, was wahr ist.\par
			(La foi, c'est refuser de connaître la vérité.)
			\par{Friedrich Nietzsche}
		\end{flushright}
\end{minipage}\end{flushright}%
\vspace*{\fill}\clearpage\thispagestyle{empty}\vfill\cleardoublepage
	
\chapter*{Remerciements}

\def\skipaknow{12pt}
\begin{flushright}
\vfill{}

Tout d'abord à mes parents, Deise et Walter, pour avoir proportionné ma vie.\\[\skipaknow]
	
À mes directeurs M. Jacky Dulcy et M. Thierry Belmonte pour leur confiance de m'avoir accepté pour réaliser cette thèse et pour tout leur soutient pendant le développement de ces travaux de thèse.\\[\skipaknow]

À l'IRT M2P pour le financement du projet à travers des partenaires industriels Safran, PSA, Peugeot Citroën, Faurecia, ECM Technologies, Ascometal, Air Liquide, Airbus Helicopters, Arcelor Mittal, UTC Aerospace Systems et Poclain Hydraulics.\\[\skipaknow]

À M. Pascal Lamesle, M. Grégory Michel et M. Simon Thibault pour leur support au cours de mes études. À Mme. Andrea Puech pour tout l'aide avec les aspects administratifs et spécialement pour sa patience avec ma \emph{distance} des aspects bureaucratiques.\\[\skipaknow]

À toute l'équipe de l'IJL, spécialement à M. Francis Kosior, pour son aide dans le monde du numérique. À M. Jaafar Ghanbaja et son inséparable microscope pour les plus belles micrographies en transmission et pour tout le partage de ses connaissances.\\[\skipaknow]

À M. Abdelkrim Redjaïmia pour nous avoir clarifié les résultats obtenus en diffraction d'électrons. À Mme. Christine Gendarme pour la réalisation des analyses chimiques des profils de diffusion et pour sa patience avec mon incapacité de polir l'emboutissage métallique.\\[\skipaknow]

À \emph{Baloo048} pour avoir proportionné des moments de simulation et programmation indescriptibles. Pour toutes les nuits que passées ensemble.\\[\skipaknow]

À mes amis Edgar Castro et Antonio Olmedilla pour être toujours disponibles pour discuter et boire un verre avec moi. À Bruno Borges Ramos pour son retard éternel à venir.\\[\skipaknow]

%À Isadora Deschamps pour m'avoir motivé de quitter ma zone de confort pour démarrer cette thèse. Pour toutes les années que je ne vais jamais oublier.\\[\skipaknow]

À Nancy Perdomo pour son amour.
\vfill{}
\end{flushright}

\cleardoublepage\phantomsection
\pdfbookmark{Resume}{Resume}
\chapter*{Résumé}

Le développement de matériaux d'ingénierie combinant ténacité et résistance à l'usure reste encore un défi. Dans le but de contribuer à ce domaine, cette thèse présente une étude de la carbonitruration des aciers 16NiCrMo13 et 23MnCrMo5. L'évolution cinétique des atmosphères à base d'hydrocarbures et d'ammoniac est étudiée numériquement, ainsi que le comportement local à l'équilibre et la cinétique de diffusion pour l'obtention de profils d'enrichissement des alliages traités. Les simulations sont confrontées à des mesures par chromatographie en phase gazeuse des produits de pyrolyse de l'acétylène et de décomposition de l'ammoniac, et aux réponses métallurgiques, par l'évaluation des profils de diffusion, des filiations de dureté et par l'identification des précipités formés par microscopie électronique en transmission. La dureté obtenue après trempe et traitement cryogénique évolue selon la racine carrée de la teneur en interstitiels en solution solide simulée à partir de la composition locale en utilisant des mesures des profils chimiques en carbone et en azote. Après revenu, les zones enrichies en azote montrent une tenue en dureté supérieure à celles obtenues avec la même teneur totale en carbone en solution, ce qui a été attribué après observation par microscopie électronique en transmission à une fine précipitation de nitrures de fer lors de cette dernière étape de traitement. Le bilan de matière des produits de pyrolyse montre que les principales espèces non détectées sont des radicaux fortement carbonés qui peuvent aussi donner lieu à la formation d'hydrocarbures polycycliques de haut poids moléculaire dans les zones froides du réacteur. À la pression atmosphérique et à basse pression l'établissement de conditions d'enrichissement en carbone à concentration constante est possible en utilisant de faibles pressions partielles d'acétylène dilué dans l'azote. La conversion atteinte par la pyrolyse de ce précurseur est pourtant importante à la température de traitement compte tenu du temps de séjour caractéristique du réacteur employé à la pression atmosphérique. La cinétique de décomposition de l'ammoniac étant beaucoup plus lente que celle des hydrocarbures légers, il a été possible de quantifier la vitesse de décomposition de cette espèce par unité de surface métallique exposée pendant la durée d'un traitement.

\par\vskip0.6cm
\noindent\textbf{Mots-clés:} Carbonitruration; Traitements thermochimiques; Martensite; Cinétique chimique; Modèles de réacteur.

\clearpage\phantomsection
\pdfbookmark{Abstract}{Abstract}
\chapter*{Abstract}

The development of engineering materials combining both toughness and wear resistance is still a challenge. Aiming to contribute to this field of study, this thesis presents a study of the carbonitriding process of alloys 16NiCrMo13~and 23MnCrMo5. Kinetics of hydrocarbon- and ammonia-based atmospheres, as well as local equilibrium and diffusion kinetics for achieving the enrichment profiles, are studied by numerical simulation. These simulations are compared to chromatography measurements of gas phase pyrolysis products of acetylene and ammonia decomposition, and with metallurgical responses, where the comparison is made with evaluated diffusion profiles, hardness measurements and the identification of precipitates by transmission electron microscopy. Hardness after quench and cryogenic treatment depends on the square root of total solid solution interstitial content simulated by using local carbon and nitrogen compositions obtained experimentally. After tempering, the regions enriched in nitrogen show better hardness stability than those with same total carbon interstitial content, what was linked to a fine precipitation of iron nitrides observed by transmission electron microscopy. Mole balance of pyrolysis products show that the main non-detected species are high-carbon radicals, which may also lead to the formation of polycyclic aromatic hydrocarbons of high molecular weight at the reactor outlet. At both atmospheric and reduced pressures, constant concentration enrichment boundary conditions were established by using low partial pressures of acetylene diluted in nitrogen. Pyrolysis of this precursor attains high conversion rates at treatment conditions given the important residence time of the atmospheric pressure reactor. Ammonia decomposition kinetics being much slower than that of low molecular weight hydrocarbons, it was possible to identify the decomposition rate of this species over a metallic sample during  a treatment. 

\par\vskip0.6cm
\noindent\textbf{Keywords:} Carbonitriding; Thermochemical treatments; Martensite; Chemical kinetics; Reactor models.

\endinput
