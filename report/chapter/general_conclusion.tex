\chapter*{Conclusion générale}

Des traitements thermochimiques de cémentation, nitruration et carbonitruration ont été réalisés sur des alliages 16NiCrMo13 et 23MnCrMo5. Du point de vue métallurgique, nous nous sommes intéressés aux mesures des profils de diffusion, aux microstructures obtenues et aux filiations de dureté ainsi produites. L'introduction du carbone et de l'azote a été réalisée en utilisant des atmosphères proches de l'équilibre pour lesquelles le contrôle des enrichissements a pu être obtenu à partir des simulations thermodynamiques. Par ailleurs, l'étude des atmosphères à base de \ch{C2H2} et de \ch{NH3} a été abordée d'un point de vue cinétique à haute température \textendash{} de l'ordre de \SI{1173}{\kelvin} \textendash{} dans le cas des traitements à basse pression.

Les études métallurgiques ont montré un bon accord entre les simulations des profils de diffusion et les paramètres opératoires qui définissent les conditions aux limites des enrichissements en carbone. Dans le cas des enrichissements en azote, ce bon accord est observé uniquement pour la nuance 23MnCrMo5, l'alliage 16NiCrMo13 étant enrichi avec une concentration en surface moins importante que celle attendue de l'équilibre gaz--solide. Cela est probablement lié à la présence de \ch{Ni} dans l'alliage, qui introduit une résistance de transfert de matière à l'interface. Les filiations de dureté observées pour la cémentation et la carbonitruration ont conduit à des résultats similaires. Cependant, des concentrations totales en interstitiels différentes ont été observées dans les couches enrichies par ces deux traitements. Pour expliquer cela, on a choisi d'adopter un modèle de durcissement~\cite{Cohen1968,Norstrom1976,Hutchinson20115845} prenant en compte uniquement le nombre d'atomes en solution solide: la contribution à la dureté de l'azote précipité sous forme de nitrures \ch{MN} a été négligée. Dans un premier temps, ce modèle s'est avéré utile pour expliquer les propriétés après trempe et traitement cryogénique. Seule la nuance 23MnCrMo5 a présenté une dureté finale après revenu supérieure en carbonitruration par rapport à celle obtenue en cémentation.

Dans le cas de la nitruration austénitique, un durcissement lors des revenus réalisés au--dessus de \SI{573}{\kelvin}, probablement lié à une précipitation de nitrures, a été mis en évidence. Cela a été étudié par microscopie électronique en transmission. On a montré la présence de deux morphologies de nitrures \ch{Fe16N2}, une déjà présente après trempe et traitement cryogénique et l'autre qui apparait uniquement après revenu. Cette morphologie observée après trempe possède des dimensions de l'ordre de \SI{100 x 10}{\nano\metre} et n'est probablement pas responsable du durcissement secondaire en raison de sa présence après nitruration. Les échantillons qui ont été revenus à \SIlist{573;673}{\kelvin} présentaient une multitude de nitrures de st{\oe}chiometrie \ch{Fe16N2} mais avec dimensions de moins de \SI{5}{\nano\metre}, pouvant être à l'origine du durcissement observé. En outre, ces études par MET ont confirmé la présence des nitrures \ch{MnSiN2} observés par \citet{Catteau2016} dans l'alliage 23MnCrMo5.

Les études expérimentales de pyrolyse de \ch{C2H2} ont correctement reproduit l'ordre de grandeur des résultats rapportés dans la littérature à basse pression. Nous avons aussi réalisé des études avec des temps de séjour beaucoup plus longs \textendash{} de l'ordre de \SI{200}{\second} \textendash{} que ceux rapportés précédemment, lesquels ont montré qu'à une température de \SI{1173}{\kelvin} moins de 30\% des atomes de carbone se trouvent dans des hydrocarbures de type \ch{CH4} et \ch{C2}. La formation d'espèces lourdes y contribue. La cémentation à pression atmosphérique réalisée à partir d'un mélange \ch{N2 - 0,02 C2H2} a conduit à des enrichissements à concentration constante mais au--dessus de la limite de solubilité simulée avec Thermo-Calc~\cite{Andersson2002,Borgenstam2000}. Dans le cas des enrichissements à basse pression, on a observé le rôle du débit, ce qui peut indiquer un mécanisme radicalaire de cémentation avec craquage de \ch{C2H2} sur les surfaces traitées, désorption et adsorption partielles des radicaux formés. L'étude de décomposition de l'ammoniac à pression atmosphérique a permis d'établir les conditions aux limites d'enrichissement en azote lors des étapes de nitruration des traitements thermochimiques: à \SI{1173}{\kelvin} environ 90\% du \ch{NH3} est dissocié dans les conditions étudiées. L'étude à basse pression de décomposition de \ch{NH3} montre que sa transformation en \ch{H2} et \ch{N2} n'est pas favorisée en raison de la faible action de masse à des pressions de l'ordre de \SIrange{50}{100}{\hecto\pascal}. La nitruration austénitique à partir de \ch{NH3} se produit apparemment selon un mécanisme radicalaire dont la recombinaison des produits de déshydrogénation partielle du précurseur reforme de l'ammoniac dans l'atmosphère.

Les études numériques de pyrolyse de l'acétylène ont montré la pertinence du mécanisme de \citet{Norinaga2009} pour la simulation des processus en phase gazeuse de décomposition de ce précurseur dans les conditions étudiées. Un bon accord entre mesures et prédictions de la fraction en \ch{C2H2} a été trouvé en utilisant un modèle de réacteur de type piston où le profil de températures mesuré dans le réacteur a été utilisé. Des ajustements réalisés grâce à nos mesures à basse pression et à pression atmosphérique ont permis d'établir des expressions globales d'ordre $n>2$. Ces expressions sont en accord avec le chemin réactionnel proposé par \citet{Norinaga2005} qui favorise la formation de benzène et dans un deuxième temps d'hydrocarbures à haut poids moléculaire. Un mécanisme simplifié de pyrolyse de \ch{C2H2} avec 40 espèces a été obtenu et il a permis une bonne prédiction de la conversion du précurseur. Dans le cas de l'ammoniac, aucun des mécanismes obtenus de la littérature n'a permis directement une simulation pertinente des processus étudiés. Les parois en quartz du réacteur semblent jouer un rôle important sur la décomposition de \ch{NH3} et nous avons proposé une expression globale utilisant l'énergie d'activation proposée par \citet{Cooper1988} pour reproduire correctement de nos résultats expérimentaux.

\chapter*{Suggestions pour l'avenir}


Pour ce qui sont des perspectives à ce travail, nous proposons un certain nombre de points listés ci--dessous:
\begin{itemize}
  \item Étude par microscopie électronique en transmission de la formation de nano-précipités dans des alliages modèles \ch{Fe - $x$ Cr - N} avec différentes teneurs en chrome et évaluation de leur effet sur la tenue en dureté après revenu.

  \item Étude par dilatométrie et calorimétrie de lames de l'alliage 16NiCrMo13 décarburées puis nitrurées pour permettre une compréhension détaillée de la formation de \ch{Fe16N2}. Cette étude devra permettre la réalisation de traitements cryogéniques directement dans l'équipement d'analyse \textemdash{} dilatomètre ou calorimètre. 
  
  \item Analyse par AP-FIM~\footnote{De l'Anglais Atom-probe field-ion microscopy.} de la localisation de l'azote \textendash{} comme cela a été réalisé pour le carbone par \citet{Hutchinson20115845} \textendash{} après trempe pour confirmer la validité de l'extension du modèle de \citet{Norstrom1976} à l'azote.
  
  \item Étude de la décomposition de l'ammoniac en couplant FTIR et chromatographie gazeuse en présence d'échantillons de différents matériaux métalliques et avec un condensateur à la sortie pour quantification éventuelle du produit liquide \ch{N2H4}.
  
  \item Conception d'un réacteur de type piston pour l'étude de la cémentation avec interruption de cycle pour une étude de prise de masse à basse pression. Cela permettrait de déterminer des conditions aux limites réalistes et de vérifier l'homogénéité du traitement le long de la direction de l'écoulement sur l'échantillon.

  \item Évaluation en fonction de la position par ToF-SIMS des radicaux adsorbés le long d'un échantillon métallique cémenté dans un réacteur de comportement piston pour différents temps d'enrichissement. Cela permettra d'approfondir les études réalisées par \citet{Kula2005}.
  
  \item Simulation et validation expérimentales à l'échelle semi--industrielle de la décomposition de \ch{C2H2} avec le mécanisme cinétique simplifié obtenu au Chapitre~\ref{ch:modelisation_cinetique}. Ces études doivent incorporer différents chargements et mesures de débits en entrée et en sortie de réacteur.
\end{itemize}

\endinput