%% References:
%% 1. http://tex.stackexchange.com/questions/14901/
%% 

\documentclass{standalone}

%%%%%%%%%%%%%%%%%%%%%%%%%%%%%% Packages
\usepackage[utf8]{inputenc}
\usepackage[T1]{fontenc}
\usepackage{lmodern} 
\usepackage[francais]{babel}
  \renewcommand*\sfdefault{phv}
  \renewcommand{\familydefault}{\sfdefault}   
\usepackage{tikz}
  \usetikzlibrary{patterns,arrows}

%%%%%%%%%%%%%%%%%%%%%%%%%%%%%% Configuration

% Adapted from the 'patterns' library: enlarged the
% distance between the lines from 4pt to 10pt

\pgfdeclarepatternformonly{north east lines wide}{
  \pgfqpoint{-1pt}{-1pt}}{\pgfqpoint{10pt}{10pt}}{\pgfqpoint{9pt}{9pt}}%
  {
    \pgfsetlinewidth{0.4pt}
    \pgfpathmoveto{\pgfqpoint{0pt}{0pt}}
    \pgfpathlineto{\pgfqpoint{9.1pt}{9.1pt}}
    \pgfusepath{stroke}
  }

\tikzset{%
  body/.style={
    inner sep=0pt,
    outer sep=0pt,
    shape=rectangle,
    draw,
    thick,
    pattern=north east lines wide
  },
  dimen/.style={
    <->,
    >=latex,
    thin,
    every rectangle node/.style={
      fill=white,
      midway,
      font=\sffamily
    }
  },
  symmetry/.style={
    dashed,
    thin
  },
}

\begin{document}
  \begin{tikzpicture}
        %\node at (0.25, 9){Réacteur à la pression atmosphérique};    
        
        %% Reactor tube
        \node [
          body,
          fill = white,
          minimum height = 0.28cm,
          minimum width  = 8.00cm,
          anchor = south west
        ] (body1) at (0.0,0.0) {};
      
        %% Dimensions
        \draw (body1.north west) -- ++(0,+2) coordinate (D1) -- +(0,-5pt);
        \draw (body1.north east) -- ++(0,+2) coordinate (D2) -- +(0,-5pt);
        \draw [dimen] (D1) -- (D2) node {0.80 m};   
        
        %% Thermal insulation
        \node [
          body, 
          pattern = north east lines,
          pattern color = gray,
          minimum height = 0.6cm,
          minimum width  = 4.0cm,
          anchor = south west
        ] (therm1) at (2.0,0.28) {};
        \node [
          body,
          pattern = north east lines,
          pattern color = gray,
          minimum height = 0.6cm,
          minimum width  = 4.0cm,
          anchor = south west
        ] (therm2) at (2.0,-0.6) {};
      
        %% Dimensions
        \draw (therm1.north west) -- ++(0,+0.6) coordinate (D1) -- +(0,-5pt);
        \draw (therm1.north east) -- ++(0,+0.6) coordinate (D2) -- +(0,-5pt);
        \draw (body1.north west) -- ++(0,+1.2) coordinate (D3) -- +(0,-5pt);
        \draw [dimen] (D1) -- (D2) node {0.40 m};
        \draw [dimen] (D3) -- (D1) node {0.20 m};
              
        %% Heater zone
        \node [
          body,
          fill = white,
          minimum height = 0.3cm,
          minimum width  = 4.0cm,
          anchor = south west
        ] (heat1) at (2.0,0.28) {};
        \node [
          body,
          fill = white,
          minimum height = 0.3cm,
          minimum width  = 4.0cm,
          anchor = south west
        ] (heat2) at (2.0,-0.3) {};
      
        %% Resistances
        \newcommand{\mylist}{
          2.25, 2.50, 2.75, 3.00, 3.25, 3.50, 3.75, 4.00,
          4.25, 4.50, 4.75, 5.00, 5.25, 5.50, 5.75}
        \foreach \x in \mylist \filldraw[black] (\x,0.43) circle (0.6mm);	
        \foreach \x in \mylist \filldraw[black] (\x,-0.15) circle (0.6mm);	
            
        %% Homogeneous zone
      	\shade[
          shading = axis,
          left color = white, 
          right color = black
        ] (body1.south west) rectangle (4.5,0.28);
      	\shade[
          shading = axis,
          left color = black, 
          right color = white
        ] (4.49,0.00) rectangle (body1.north east);
        
        %% Pressure gauge
        \node [
          body,
          fill = gray,
          very thin,
          minimum height = 0.2cm,
          minimum width  = 0.05cm,
          anchor = south west
        ] (gauge) at (7.0,0.28) {};

        \filldraw[fill=gray] (7.025,0.58) circle (0.2);
        \node at (7.025,0.58) {\small P}; 

        \draw (body1.north east) -- ++(0.1,0.00) coordinate (D1) -- 
          +(0,+1cm) node[below=10pt, rotate=90] {0.028 m};
        \draw (body1.south east) -- ++(0.1,0.00) coordinate (D2) -- +(0,-8pt);
        \draw [dimen,-] (D1) -- (D2);
        \draw [dimen,<-] (D1) -- ++(0,+5pt);
        \draw [dimen,<-] (D2) -- ++(0,-5pt);
       
        %% Draw symmetry lines
        \draw [symmetry, very thin] (-0.25,0.14) -- (8.25,0.14);
  \end{tikzpicture}
\end{document}